\subsection{要件仕様と設計仕様書の相互確認}
要件仕様と設計仕様において誤りや抜けがないかどうかについて
確認し,設計仕様についても細かく設計できていることを確認した.
\subsection{関数ごとの単体テスト}
関数毎の機能が設計仕様を満たしていることを確認した.
\subsection{全体のシステムテスト}
以下のテスト項目に従ってシステムの確認を行った.
1回目のテストで不具合を発見したため,仕様の確認を行い
その後2回目のテストを行った.2回目のテストでは,全ての項目が
正常な動作であった.
\begin{table}[htb]
\begin{tabular}{|l|l|l|}
\hline
テスト内容                             & 1回目 & 2回目 \\ \hline
落ちてくるブロックがランダムであるか                & 正常  & 正常  \\ \hline
ブロックが7種類全て出現するか                   & 正常  & 正常  \\ \hline
ブロックがフィールドを出ないか                   & 正常  & 正常  \\ \hline
ブロックの操作(下左右移動ができるか)               & 正常  & 正常  \\ \hline
操作中のブロックが固定されるか                   & 正常  & 正常  \\ \hline
ブロック固定後に次のブロックが落ちてくるか             & 正常  & 正常  \\ \hline
ブロックの落ちてくる速度が1{[}block/sec{]}であるか & 正常  & 正常  \\ \hline
回転によるはめ込み等の複雑な処理が実装されていないか        & 正常  & 正常  \\ \hline
ブロックが1行揃ったら消えるか                   & 正常  & 正常  \\ \hline
ブロックが消えたら1行詰められているか               & 正常  & 正常  \\ \hline
ポイントアップがルール通りに行われているか             & 異常  & 正常  \\ \hline
\end{tabular}
\end{table}
