\subsection{プレイした人の感想}
\begin{itemize}
  \item ブロックが1色じゃないほうが楽しい.
  \item 次に落ちてくるブロックが分かるほうが使いやすい.
  \item ホールド機能が欲しい.
  \item スコアによってスピードが変化するほうが良い.
\end{itemize}

\subsection{改善点}
\begin{itemize}
  \item ブロックを多色にする.
  \item 次に落ちてくるブロックの表示を行う.
  \item ホールド機能を実装する.
  \item 速度の変化を実装する.
\end{itemize}

\subsection{設計・作成に関する反省}
今回は,ウォーターフォール設計を用いてソフトウェアを作成した.
1つ1つの工程で成果が文章化出来た点では,資料作成や日程の管理がやりやすかった.
しかし,バグを発見した際に仕様まで戻る必要があったため予想以上の時間がかかった.
また,最終形にならないとシステムの形が見えてこないので仕様の変更(ユーザーの要求)
には対応しづらいと思った.
